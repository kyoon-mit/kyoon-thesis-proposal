\documentclass{article}

\usepackage{cite}% cite with hyphen
\usepackage[breaklinks, colorlinks=true, linkcolor=black, citecolor=black, urlcolor=black]{hyperref}
\usepackage{sty/hepparticles}
\usepackage{sty/heppennames2}

%opening
%\title{}
%\author{}

\begin{document}

%\maketitle

Since the discovery of the Higgs boson in 2012 by the ATLAS and CMS collaborations at the Large Hadron Collider (LHC) \cite{2012_ATLAS_Higgs, 2012_CMS_Higgs, 2013_CMS_Higgs}, the observed values of its Yukawa couplings to fermions have been in agreement with the Standard Model (SM) predictions. Experimental confirmations include coupling to top quarks through associated production \cite{2018_ATLAS_ttH, 2018_CMS_ttH}, to \(\PGt\) leptons \cite{2018_CMS_H_tautau, 2019_ATLAS_H_tautau}, and to bottom-quarks \cite{2018_ATLAS_H_bb, 2018_CMS_H_bb} through direct decays. ... (Charm quark and light quarks.)

Theory 1. Higgs radiative decays. Mechanism. Advantageous because... Expected BR.\\
Theory 2. Charm-quark fragmentation. Same format. Compare with charm radiative decay.\\
An 1. Started in 2021, published 2024. Summarize abstract. Compare with ATLAS. Significance of this research.\\
An 2. Started in 2024. Steps that have been done so far. Steps needed to be taken. Data from Run 3 schedule.\\
My plans to be at CERN in June 2025. Expected graduation time range.


\pagebreak
\bibliographystyle{JHEP-kyoon}
\bibliography{refs.bib}

\end{document}
