\documentclass{article}

\usepackage[letterpaper, margin=1in]{geometry}

\usepackage{cite}% cite with hyphen
\usepackage[breaklinks, colorlinks=true, linkcolor=black, citecolor=black, urlcolor=black]{hyperref}
\usepackage{sty/hepparticles}
\usepackage{sty/heppennames2}

%opening
\title{MIT Department of Physics PhD Thesis Proposal}
\author{Kyungseop Yoon}

\begin{document}

\maketitle

Since the discovery of the Higgs boson in 2012 by the ATLAS and CMS collaborations at the Large Hadron Collider (LHC) \cite{2012_ATLAS_Higgs, 2012_CMS_Higgs, 2013_CMS_Higgs}, the observed values of its Yukawa couplings to fermions have been in agreement with the Standard Model (SM) predictions \cite{2022_ATLAS_Higgs_10yrs, 2022_CMS_Higgs_10yrs}. Experimental confirmations include the couplings to top quarks through associated production \cite{2018_ATLAS_ttH, 2018_CMS_ttH}, to \(\PGt\) leptons \cite{2018_CMS_H_tautau, 2019_ATLAS_H_tautau}, and to bottom-quarks \cite{2018_ATLAS_H_bb, 2018_CMS_H_bb} through direct decays. The amplitudes of the couplings are linearly proportional to the fermion masses, limiting the sensitivity to second and first generation quarks and leptons. ... (Muons, charm quark, and light quarks.)\\

%chatgpt answer
My thesis will investigate the decays of the Higgs boson into light-quark mesons and charmonium states, with the goal of probing the Higgs couplings to first and second-generation quarks. If the observed coupling strengths align with theoretical predictions, it will support the validity of the Standard Model (SM). Deviations, however, could indicate the need for modifications to the electroweak theory. However, given the available data within an estimated time frame of my PhD program, it is more realistic to focus on setting upper limits on the branching ratios (BR) of the decay channels and, by extension, the coupling strengths.\\

\vspace{5em}
Theory 1. Higgs radiative decays. Mechanism. Advantageous because... Expected BR.\\
Theory 2. Charm-quark fragmentation. Same format. Compare with charm radiative decay.\\
An 1. Started in 2021, published 2024. Summarize abstract. Compare with ATLAS. Significance of this research.\\
An 2. Started in 2024. Steps that have been done so far. Steps needed to be taken. Data from Run 3 schedule.\\
My plans to be at CERN in June 2025. Expected graduation time range.


\pagebreak
\bibliographystyle{JHEP-kyoon}
\bibliography{refs.bib}

\end{document}
