\documentclass{article}

\usepackage[letterpaper, margin=1in]{geometry}

\usepackage{cite}% cite with hyphen
\usepackage[breaklinks, colorlinks=true, linkcolor=black, citecolor=black, urlcolor=black]{hyperref}
\usepackage{sty/hepparticles}
\usepackage{sty/heppennames2}

%opening
\title{MIT Department of Physics PhD Thesis Proposal}
\author{Kyungseop Yoon}

\begin{document}

\maketitle

Since the discovery of the Higgs boson in 2012 by the ATLAS and CMS collaborations at the Large Hadron Collider (LHC) \cite{2012_ATLAS_Higgs, 2012_CMS_Higgs, 2013_CMS_Higgs}, the observed values of its Yukawa couplings to fermions have been in agreement with the Standard Model (SM) predictions \cite{2022_ATLAS_Higgs_10yrs, 2022_CMS_Higgs_10yrs}. Experimental confirmations include the couplings to top quarks through associated production \cite{2018_ATLAS_ttH, 2018_CMS_ttH}, to \(\PGt\) leptons \cite{2018_CMS_H_tautau, 2019_ATLAS_H_tautau}, and to bottom-quarks \cite{2018_ATLAS_H_bb, 2018_CMS_H_bb} through direct decays. The amplitudes of the couplings are linearly proportional to the fermion masses, limiting the sensitivity to second and first generation quarks and leptons. ... (Muons, charm quark, and light quarks.)\\

%chatgpt answer
My thesis will investigate the decays of the Higgs boson into light-quark mesons and charmonium states, with the goal of probing the Higgs couplings to first and second-generation quarks. If the observed coupling strengths align with theoretical predictions, it will support the validity of the Standard Model (SM). Deviations, however, could indicate the need for modifications to the electroweak theory. Given the available data within an estimated time frame of my PhD program, it is more realistic to focus on setting upper limits on the branching ratios (BR) of the decay channels and, by extension, the coupling strengths.\\

Rather than search for direct decays of the Higgs to the first and second-generation quarks (\(\PH\rightarrow\PQq\PAQq\)), it is more advantageous to look at the mentioned alternative channels for two reasons. First, the large QCD background severely limits the sensitivity to the direct searches. Second, the low Yukawa couplings can be more sensitively probed by tagging a boosted \(\PGg\) or a \(\PJGy\).\\

For that purpose, I propose two analyses of data from the CMS experiment at LHC. The first analysis probes the Yukawa couplings of the Higgs to u, d, and s quarks. The set of decays that we explore are Higgs radiative decays to a light-quark meson and a photon. The meson is a phi meson, rho meson, or a \(\PK^{*0}\) meson. The phi meson is a bound state of strange quarks, the rho meson is a bound state of up and down quarks, and the \(\PK^{*0}\) meson is a bound state of down and strange quarks. For the quark flavor conserving decays to the \(\PGf\) or \(\PGr\)(770) meson, there are two processes that interfere destructively. The direct contribution has the Higgs coupling via Yukawa directly to the quarks, where one of the quarks then radiates a photon. The indirect contribution has an off-shell photon or \(\PZ\) boson produced in \(\PH\rightarrow\PGg\PGg^*, \PZ\PZ^*\) which then fragments into a meson. In the flavor violating probe to the \(\PK^{*0}\) meson, the Higgs boson couples indirectly to the s and d quarks indirectly via mediation with top quarks or \(\PW\) bosons. The predicted Standard Model branching ratio for the processes \(\PH\rightarrow\PGf\gamma\), \(\PH\rightarrow\PGr\gamma\), and \(\PH\rightarrow\PK^{*0}\gamma\) are \(\mathcal{O}(10^{-5})\), \(\mathcal{O}(10^{-6})\), and \(< \mathcal{O}(10^{-11})\), respectively.\\

The second analysis probes the Yukawa couplings to the c quark via two charm-jets and a di-muon final state from a \(\PJGy\) meson. The Higgs couples first to two charm quarks, whereby one of the charm quark fragments into a pair of charm quarks. Among the four charm quarks, two are bounded to form a \(\PJGy\) meson and two become jets. The clean decay signature of \(\PJGy\rightarrow\PGmp\PGmm\) provides an efficient tagger. The branching fraction of \(\PH\rightarrow\PJGy + \PQc\PAQc\) is about \(2\times10^{-5}\).\\

The plans for the two analyses are as follows. (Anything interesting to say here besides that bland sentence?) The first analysis began in 2021 and was published in 2024 with the full Run 2 data (lumi?) from the CMS experiment. It found upper limits to the branching ratio of (something), which are the most stringent upper limits to date. Compared to this, the ATLAS experiment published limits of (smth) in 2017 (???) with (lumi?). It was presented at the 2024 Moriond Conference (exact name) and has been submitted to Physics Review Letters B for review in October (??). Summary of the analysis??.\\

The second analysis began in 2024. It will include the all the data from Run 2 and part or all of the data from Run 3. (Triggers.) What are the steps involved?\\

I plan to relocate to CERN in June, 2025 for my PhD research. The primary benefit is the immediate collaboration with the researches involved in the \(\PH\rightarrow\PJGy + \PQc\PAQc\) analysis. I will stay there until my graduation, which will be possible once the analysis is published. The progress of my analysis is largely constrained by the data-taking period of Run 3. Depending on the decision of the collaboration, I will have to include either part of or all of the Run 3 data for my analysis.\\

(Conclusion.)


\vspace{5em}
Theory 1. Higgs radiative decays. Mechanism. Advantageous because... Expected BR.\\
Theory 2. Charm-quark fragmentation. Same format. Compare with charm radiative decay.\\
An 1. Started in 2021, published 2024. Summarize abstract. Compare with ATLAS. Significance of this research.\\
An 2. Started in 2024. Steps that have been done so far. Steps needed to be taken. Data from Run 3 schedule.\\
My plans to be at CERN in June 2025. Expected graduation time range.


\pagebreak
\bibliographystyle{JHEP-kyoon}
\bibliography{refs.bib}

\end{document}
