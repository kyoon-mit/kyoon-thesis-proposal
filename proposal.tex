\documentclass{article}

\usepackage[letterpaper, margin=1in]{geometry}

\usepackage{cite}% cite with hyphen
\usepackage[breaklinks, colorlinks=true, linkcolor=black, citecolor=black, urlcolor=black]{hyperref}
\usepackage{sty/hepparticles}
\usepackage{sty/heppennames2}

\newcommand{\rhomeson}{\ensuremath{\PGr\mathrm{(770)}^0}}
\newcommand{\phimeson}{\ensuremath{\PGf\mathrm{(1020)}}}
\newcommand{\Kstarmeson}{\ensuremath{\PK^*\mathrm{(892)}^0}}
\newcommand{\Hjpsicc}{\ensuremath{\PH\rightarrow\PJGy + \PQc\PAQc}}

%opening
\title{PhD Thesis Proposal}
\author{Kyungseop Yoon}
\date{%
	PhD Candidate, MIT Department of Physics\\[2ex]%
	\today
}

\begin{document}

\maketitle

Since the discovery of the Higgs boson in 2012 by the ATLAS and CMS collaborations at the Large Hadron Collider (LHC) \cite{2012_ATLAS_Higgs, 2012_CMS_Higgs, 2013_CMS_Higgs}, a series of studies have been confirming the predicted values of its Yukawa couplings to fermions in the Standard Model (SM) \cite{2022_ATLAS_Higgs_10yrs, 2022_CMS_Higgs_10yrs}. Experimental confirmations include the couplings to top quarks through associated production \cite{2018_ATLAS_ttH, 2018_CMS_ttH}, to \(\PGt\) leptons \cite{2018_CMS_H_tautau, 2019_ATLAS_H_tautau}, and to \(\PQb\) quarks \cite{2018_ATLAS_H_bb, 2018_CMS_H_bb}. The amplitudes of the couplings are linearly proportional to the fermion masses, limiting the sensitivity to first and second generation quarks and leptons.\\
%The branching fractions compared to top quarks are etc. etc. good sentence... (Muons, charm quark, and light quarks.) (https://link.springer.com/article/10.1007/JHEP01(2021)148)\\

%chatgpt answer
The aim of my thesis is to probe the Higgs couplings to first and second generation quarks. If the observed coupling strengths align with theoretical predictions, it will support the validity of the Standard Model (SM). Deviations, however, could indicate the need for modifications to the electroweak theory. Given the available data within an estimated time frame of my PhD program, it is more realistic to place upper limits on the branching fractions (BR) of the decay channels rather than to expect an observation of an excess of signal.\\

There are several challenges facing the direct searches for the decays of the Higgs to first and second generation quarks (\(\PH\rightarrow\PQq\PAQq\)). First, the large QCD background limits the sensitivity in direct searches. While the CMS and ATLAS collaborations have announced results on direct searches for charm quarks (\(\PH\rightarrow\PQc\PAQc\)) \cite{2022_ATLAS_H_cc, 2023_CMS_H_cc}, the studies rely on the small production cross section of \(\Pp\Pp\rightarrow\mathrm{V}(\rightarrow\Pl\Pl,\,\Pl\PGnl)\PH\) and novel jet flavor identification algorithms to limit the QCD background. Second, the Yukawa couplings of the Higgs boson to lighter quarks decrease proportionally with their masses. As a result, the branching fractions of \(\PH\rightarrow\PQc\PAQc\) are 20 times smaller than that of \(\PH\rightarrow\PQb\PAQb\), and the branching fractions of \(\PH\rightarrow\PQs\PAQs,\,\PQu\PAQu,\,\PQd\PAQd\) are even less \cite{CERN_report4}. Finally, the reconstruction of the final states itself is a challenge due to the hadronization of the quarks. Due to these difficulties, alternative search channels have been proposed.\\

For my thesis research, I propose to analyze two alternative search channels. The first analysis is of the type, \(\PH\rightarrow\mathrm{M}\gamma\), where \(\mathrm{M}\) is a light-quark meson among one of the following: \rhomeson{}, \phimeson{}, or \Kstarmeson{} \cite{2015_H_Mgamma_theory}. The \rhomeson{} is a bound state of \(\PQu\), \(\PQd\) quarks and antiquarks, and the \phimeson{} is a bound state of \(\PQs\) quark and antiquark. Hence, the search channel probes the couplings to the \(\PQu\), \(\PQd\), and \(\PQs\) quarks. In the flavor-conserving decays to the \rhomeson{} or \phimeson{}, there are two processes that interfere destructively. One process involves a direct Higgs Yukawa coupling to the quarks, whereas the other couples indirectly via an off-shell photon or \(\PZ\) boson. It is the direct contribution that is sensitive to modifications of the Yukawa couplings. The \Kstarmeson{} is a bound state of (\(\PQd\), \(\PAQs\)) or (\(\PAQd\), \(\PQs\)), which is a probe to the flavor-violating coupling of the Higgs. It is included in the analysis because it has the charged di-track final state similar to that of the flavor-conserving probes. The predicted Standard Model branching ratios for the processes \(\PH\rightarrow\PGf\gamma\), \(\PH\rightarrow\PGr\gamma\), and \(\PH\rightarrow\PK^{*0}\gamma\) are \(\mathcal{O}(10^{-5})\), \(\mathcal{O}(10^{-6})\), and \(< \mathcal{O}(10^{-11})\), respectively. \\

The second analysis probes the couplings to charm quarks via \Hjpsicc{}. The Higgs couples to two \(\PQc\) quarks, which further produces a total of four intermediary \(\PQc\) quarks via the charm-quark fragmentation mechanism. Two are bounded to form a \(\PJGy\) meson and two undergo hadronization as jets \cite{2022_H_jpsicc_theory}. The clean decay signature of \(\PJGy\rightarrow\PGmp\PGmm\) reduces the QCD background. Unlike the direct search via \(\PH\rightarrow\PQc\PAQc\), this channel does not rely on the \(\mathrm{V}\PH\) production. Instead, it can take advantage of the \(\Pg\Pg\PH\) production mode, which has a cross section of \(\mathcal{O}(10^2)\) times that of the \(\mathrm{V}\PH\) mode. The branching fraction of \(\PH\rightarrow\PJGy + \PQc\PAQc\) is about \(2\times10^{-5}\). With the Run 2 integrated luminosity at around \(150\;\mathrm{fb}^{-1}\) and a projected Run 3 integrated luminosity between \(250\) and \(300\;\mathrm{fb}^{-1}\), it is expected that roughly 25 signal events are produced during the Run 2 and 3 combined data-taking period at the CMS experiment.\\


The first analysis began in 2021 and was published in 2024 with the full Run 2 data from the CMS experiment \cite{2024_HIG_23_005}. It found exclusive upper limits at 95\% CL to the branching fractions of \(\PH\rightarrow\mathrm{M}\gamma\). The limits for \(\mathrm{M}=\rhomeson\), \(\phimeson\), and \(\Kstarmeson\) are \(3.7\times10^{-4}\), \(3.0\times10^{-4}\), and \(3.1\times10^{-4}\), which are the most stringent upper limits to date. The paper has been submitted to Physics Review Letters B. Therefore, the remainder of my thesis research will be on the second analysis.\\

The analysis of the search for \Hjpsicc{} began in 2024. As of now, we are studying the Monte Carlo (MC) simulation samples for signal and background with the conditions for 2018, the final year of Run 2. The major background events come from an inclusive \(\PJGy\) production in association with light \(\Pg\), \(\PQq\) jets, \(\Pb\rightarrow\PJGy(\rightarrow\PGmp\PGmm)+X\) decays with a displaced di-muon vertex, and contamination from the \(\PH\rightarrow\PJGy + \PQb\PAQb\) decays. The progress which we have made in our analysis so far consists of reconstructing the signal events and reducing the background through optimization of selection cuts. The cuts on the variables include the displacement of the di-muon vertex in the signal as well as the angular distribution of the final states. However, the invariant mass of the reconstructed Higgs candidate has a poor resolution. As a first step to resolving this problem, we have removed from the charm jets the muons which originate from the \(\PJGy\). In 2025, we plan to recalibrate the charm-tagging algorithm and recalculate the jet energy scale correction (JEC). These steps are necessary in order to ensure sensitivity that is comparable to the direct search. As of now, the existing charm-tagging algorithm is optimized to find a charm jet of higher momentum, and does not provide any meaningful discrimination between signal and background events in our analysis. While it is possible to reconstruct the signal events without the usage of any charm-tagger, a detailed study into the optimization of a charm-tagger can potentially lead to a greater sensitivity. The performance of the charm-tagger can be calibrated with the \(\PW\PW\) samples, where one \(\PW\) decays leptonically and the other decays to a pair of \(\PQq\) (\(\PQu, \PQd, \PQs, \PQc, \PQb\)) jets. It is projected that a majority of the analysis effort will be spent on the optimization of the charm-tagger.\\

I plan to relocate to CERN in June, 2025 for my thesis research. The primary benefit is that I have immediate access to the collaborators involved in the \(\PH\rightarrow\PJGy + \PQc\PAQc\) analysis. I will stay there until my graduation, which will be possible once the analysis successfully undergoes the collaboration-wide review process. The approximate date of the publication depends on many factors, among which is the amount of data from Run 3 the collaboration deems necessary for the analysis. The Run 3 period of data-taking is scheduled to be finished by May 2026 for p-p collisions. Hence, it is possible, or even likely, that my graduate will not take place until after the Fall Semester of 2026.\\

My thesis research will contribute to further advancing our knowledge of the Higgs Yukawa couplings to first and second generation quarks. This area of knowledge is essential to completing our verification of the electroweak theory of the Standard Model, and is challenging due to the rarity of the decays. The searches via Higgs radiative decays and charm-fragmentation will either complement the direct searches or compensate for the lack thereof. With the available data from the CMS experiment at the LHC, we expect to place stringent limits on the branching fractions of these decays, and by extension, on the Yukawa coupling modifiers.

\pagebreak
\bibliographystyle{JHEP-kyoon}
\bibliography{refs.bib}

\end{document}
