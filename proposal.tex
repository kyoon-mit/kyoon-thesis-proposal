\documentclass{article}

\usepackage[letterpaper, margin=1in]{geometry}

\usepackage{cite}% cite with hyphen
\usepackage[breaklinks, colorlinks=true, linkcolor=black, citecolor=black, urlcolor=black]{hyperref}
\usepackage{sty/hepparticles}
\usepackage{sty/heppennames2}
\usepackage{enumitem}
%\usepackage{indentfirst}

\newcommand{\rhomeson}{\ensuremath{\PGr\mathrm{(770)}^0}}
\newcommand{\phimeson}{\ensuremath{\PGf\mathrm{(1020)}}}
\newcommand{\Kstarmeson}{\ensuremath{\PK^*\mathrm{(892)}^0}}
\newcommand{\Hjpsicc}{\ensuremath{\PH\rightarrow\PJGy + \PQc\PAQc}}

%opening
\title{PhD Thesis Proposal}
\author{Kyungseop Yoon}
\date{%
	PhD Candidate, MIT Department of Physics\\[2ex]%
	\today
}

\begin{document}

\maketitle
\subsubsection*{Introduction}
Since the Higgs boson discovery in 2012 by the ATLAS and CMS collaborations at the Large Hadron Collider (LHC) \cite{2012_ATLAS_Higgs, 2012_CMS_Higgs, 2013_CMS_Higgs}, extensive studies have validated the Standard Model (SM) predictions for its Yukawa couplings to fermions  \cite{2022_ATLAS_Higgs_10yrs, 2022_CMS_Higgs_10yrs}. Experimental evidence confirms these couplings with top quarks via associated production \cite{2018_ATLAS_ttH, 2018_CMS_ttH}, with \(\PGt\) leptons \cite{2018_CMS_H_tautau, 2019_ATLAS_H_tautau}, and with \(\PQb\) quarks \cite{2018_ATLAS_H_bb, 2018_CMS_H_bb}. However, the sensitivity to the Yukawa couplings of first and second generation quarks are suppressed due to their proportionality to fermion masses, leaving an incomplete picture of the Higgs boson within the SM framework.\\
%The branching fractions compared to top quarks are etc. etc. good sentence... (Muons, charm quark, and light quarks.) (https://link.springer.com/article/10.1007/JHEP01(2021)148)\\

This thesis aims to probe the Higgs couplings to first and second generation quarks. Consistency between experimental observations and predictions would further validate the SM, while deviations could suggest necessary modifications to electroweak theory. Given the statistical limitations of the data expected to be available during the duration of my PhD program, this research will primarily focus on setting upper bounds for the branching fractions of relevant Higgs decay channels, rather than detecting a statistically significant signal excess.

\subsubsection*{Challenges in Direct Searches}
Directly searching for Higgs decays into first and second generation quarks (\(\PH\rightarrow\PQq\PAQq\)) faces three major challenges. First, the significant QCD background limits sensitivity. While the CMS and ATLAS collaborations have reported results for charm-quark decays (\(\PH\rightarrow\PQc\PAQc\)) \cite{2022_ATLAS_H_cc, 2023_CMS_H_cc}, these rely on rare processes like \(\Pp\Pp\rightarrow\mathrm{V}(\rightarrow\Pl\Pl,\,\Pl\PGnl)\PH\) and advanced jet flavor-identification algorithms to mitigate QCD interference. Second, the Yukawa couplings of lighter quarks decrease proportionally with their masses, reducing branching fractions by orders of magnitude compared to heavier quarks \cite{CERN_report4}. For instance, the branching fraction for \(\PH\rightarrow\PQc\PAQc\) is 20 times smaller than \(\PH\rightarrow\PQb\PAQb\), while \(\PH\rightarrow\PQs\PAQs\), \(\PQu\PAQu\), and \(\PQd\PAQd\) are even less. Finally, reconstructing final states remains difficult due to quark hadronization, necessitating innovative approaches to improve sensitivity.

\subsubsection*{Proposed Research}
To address the challenges in direct searches for Higgs decays to first and second generation quarks, this thesis explores two alternative channels.

\begin{enumerate}[label=\textbf{\arabic*})]
	\item \textbf{Radiative Decays \(\PH\rightarrow\mathrm{M}\gamma\)}
	
	\noindent The first approach investigates radiative decays of the Higgs boson, where M is a light-quark meson such as \rhomeson{}, \phimeson{}, and \Kstarmeson{} \cite{2015_H_Mgamma_theory}. These mesons provide a probe into the Yukawa couplings to \(\PQu\), \(\PQd\), and \(\PQs\) quarks.
	\begin{itemize}
		\item \rhomeson{} (\(\PQu\), \(\PQd\) quark bound state) and \phimeson{} (\(\PQs\) quark bound state) test flavor-conserving decays. There are two processes that interfere destructively. One process involves a direct Higgs Yukawa coupling to the quarks, whereas the other couples indirectly via an off-shell photon or \(\PZ\) boson. It is the direct contribution that is sensitive to modifications of the Yukawa couplings.
		\item \Kstarmeson{} (\(\PQd\)\(\PAQs\) or \(\PAQd\)\(\PQs\) bound state) test flavor-violating decays. The quarks couple indirectly to the Higgs boson via an off-shell photon or a \(\PZ\) boson.
	\end{itemize}
	
	\noindent The expected SM  branching ratios are \(\mathcal{O}(10^{-5}) \; (\PGf\gamma)\), \(\mathcal{O}(10^{-6}) \; (\PGr\gamma)\), and \(< \mathcal{O}(10^{-11}) \; (\PK^{*0}\gamma)\).
	
	\item \textbf{Higgs Decay to Charmonium \Hjpsicc{}}
	
	\noindent The second approach investigates Higgs decays involving charmonium production \Hjpsicc{}, where the Higgs couples to two charm quarks, which subsequently fragment into:
	\begin{itemize}
		\item A \(\PJGy\) meson (\(\PQc\PAQc\)), identifiable through its clean decay signature (\(\PJGy\rightarrow\PGmp\PGmm\)), and
		\item Two charm jets via hadronization.
	\end{itemize}
	
	Unlike direct \(\PH\rightarrow\PQc\PAQc\) searches, this channel leverages the gluon-gluon fusion \(\Pg\Pg\PH\) production mode, which has a cross section of \(\mathcal{O}(10^2)\) times higher than the \(\mathrm{V}\PH\) mode. The branching fraction is predicted to be approximately \(2\times10^{-5}\), yielding an estimated 25 signal events in combined Run 2 and Run 3 datasets with an integrated luminosity of approximately \(400\)-\(450\;\mathrm{fb}^{-1}\).
\end{enumerate}

\subsubsection*{Current Progress and Future Plans}

The radiative decay analysis began in 2021 and culminated with the publication of upper limits on \(\PH\rightarrow\mathrm{M}\gamma\) branching fractions in 2024 \cite{2024_HIG_23_005}. The limits for \(\mathrm{M}=\rhomeson\), \(\phimeson\), and \(\Kstarmeson\) are \(3.7\times10^{-4}\), \(3.0\times10^{-4}\), and \(3.1\times10^{-4}\), which are the most stringent upper limits to date. The paper has been submitted to Physics Review Letters B for review. Moving forward, the focus will shift to the \Hjpsicc{} analysis, which is currently in the signal reconstruction and background reduction phase using Run 2 Monte Carlo (MC) samples.\\

Key challenges in this analysis include:
\begin{itemize}
	\item High QCD background contamination from \(\PJGy\) production associated with light quark and gluon jets.
	\item Poor resolution of the reconstructed Higgs candidate mass due to charm-tagging inefficiencies.
\end{itemize}

\noindent To address these, future work will recalibrate charm-tagging algorithms and optimize selection cuts. Efforts will also focus on improving the invariant mass resolution by excluding \(\PJGy\)-originated muons from charm jet reconstruction. By 2025, recalibration of jet energy corrections and detailed charm-tagger studies are expected to improve sensitivity significantly.
 
\subsubsection*{CERN Relocation}

I plan to relocate to CERN in June 2025 to facilitate collaboration on the \Hjpsicc{} analysis. This move will provide immediate access to critical data and expertise, allowing for efficient execution of the research plan. Analysis progress will depend on the CMS collaboration's data-taking schedule, with the Run 3 period expected to conclude in May 2026.

\subsubsection*{Impact and Conclusion}

This thesis will contribute to advancing our understanding of Higgs Yukawa couplings to first and second generation quarks, complementing or compensating for the limitations of direct searches. By setting stringent upper bounds on branching fractions using innovative search channels, this research will help refine the SM or identify potential pathways for new physics.

\pagebreak
\bibliographystyle{JHEP-kyoon}
\bibliography{refs.bib}

\end{document}
